The study of the drunk driving simulator was conducted with 15 participants and lasted for a maximum of 30 minutes.
All the study subjects were in the process of getting their driving license for regular passenger cars and were between 15 and 17 years old.
4 out of 15 of the participants were male.
The study was held in a driving school in Tamsweg, an Austrian town in the state of Salzburg.
The persons were recruited while attending the theoretical course in the driving school and participated voluntarily in the experiment.
\\
During the testing a questionnaire (Appendix \ref{appendix:documents}) was handed out.
The first page consists of question about demographics and a self-assessment regarding driving skills, experience with VR, simulations and alcohol.
All the questions required the test subjects to answer according to a Likert-type scale \autocite[]{likert1932technique} with five items.
Each level represents a numerical value and is distributed as follows:
\begin{itemize}
    \item (1) Disagree
    \item (2) Slightly disagree
    \item (3) Neutral
    \item (4) Slightly agree
    \item (5) Agree
\end{itemize}
60\% already owned a different license and were allowed to drive a moped or a tractor in public traffic.
Many of the subjects claimed to have some experience in driving vehicles (Mean 2.87; SD 1.45) and perceived themselves as reasonable drivers (Mean 3.13; SD 0.88).
Very few had a former experience with VR applications (Mean 1.67; SD 0.79) or driving simulations in general (Mean 2.2; SD 1.16).
When asked about the quality of their education regarding alcohol consumption and effects (Mean 4.53; SD 1.02) and their personal experience with alcohol consumption (Mean 4.67; SD 0.6), almost all participants claimed to be very experienced.
The remaining pages of the questionnaire evaluated the perceived quality of the simulation and its perceived use as an education tool and are analysed in section \ref{subsection:simulator evaluation}.
\\
The experiment was designed as a "within-subject" evaluation. \autocite[]{charness2012experimental} 
Each of the participants was tasked to perform the same set of objectives two times under different circumstances.
One set of tasks, also named "run" in the following sections, had to be performed with simulated drunkenness and the other without any effects.
A single run consists of two laps in the continuous track without any interruption, with the tracking and reaction test active at the same time.
The participants were split into two groups (A and B) where group A started its first run in a sober state and vice versa.
The performance of the test subjects was tracked and the individual differences in their behaviour under the changing conditions is evaluated.
\\
Before the study commenced the participants were asked to sign a declaration of consent (Appendix \ref{appendix:documents}) and were informed about the course and goal of the study.
The test person was then familiarized with the controls of the simulation and the VR HMD.
For a maximum of ten minutes the person was tasked to drive in the virtual testing environment with all the study-relevant tests deactivated.
During this phase the drunk effects were shown and explained to the participants.
When the test subject was ready to proceed to the next phase, the actual experiment was started.
\\
Depending on the group the test person began with a drunk or sober run and was tasked to stop after a verbal signal of the test coordinator.
The time to finish was measured by test coordinator with a common stopwatch and was noted down by hand.
After a short break the second run commenced. 
During each run the participants were frequently asked for any uncomfortable sensations like nausea or dizziness.
After finishing both runs the before-mentioned questionnaire was handed out.
\\
The sections \ref{subsection:evaluation reaction} and \ref{subsection:evaluation tracking} evaluate the data collected during the experiment and analyse the impact of the drunk driving simulator on the user abilities.
To determine the statistical significance of the data, a paired t-test on each test was conducted. \autocite[]{hsu2005paired}
To apply this test a normal distribution for the value deviation within the subjects is required and was determined with the Shapiro-Wilk method. \autocite[]{razali2011power} 
For the t-test a standard alpha level was chosen ($\alpha=0.05$), the independent variable in each analysis was the drunkenness.
In section \ref{subsection:results} the results of the evaluation are summarized and examined whether they support the assumed intention of the simulation.

\subsection{Tracking test}
\label{subsection:evaluation tracking}
As in section \ref{subsection:lateral position} established, the average deviation from the reference path is calculated to determine a potential deterioration in the drivers abilities.
Different to the approach in the LCT \autocite[]{iso201026022} the measurement in this is study is done on fixed positions instead of time based sampling rates. 
For this reason no longitudinal information is required and the equation (\ref{equation:LCTMeanDeviation}) can be simplified to a common mean value calculation: 
\begin{equation}
\label{equation:SimpleMeanDeviation}
	\frac{1}{S}\sum(x_{deviation},i)
\end{equation}
To calculate the deviation (equation \ref{equation:deviation}), the data collected during the test run without any drunk effects is used as reference.
2 of the 15 participants were unable to finish the drunk run because of nausea.
The data of lateral positioning is still evaluated but only as far as the subject progressed in the course.
Their time to finish was not evaluated.

\begin{figure}[h]
    \centering
	\includegraphics[width=1\linewidth]{images/boxplot_tracking.png}
	\caption[
		Box plot tracking mean deviations
	]{
		Box plot of the tracking mean deviations from the ideal path in cm.
	}
	\label{figure:boxplotTracking}
\end{figure}


\begin{table}[]
\centering
\begin{tabular}{lll}
\multicolumn{2}{l}{Tracking mean deviations in cm - paired t-test}                       &                \\
\multicolumn{1}{c|}{\textit{}}                    & \multicolumn{1}{l|}{\textit{normal}} & \textit{drunk} \\ \hline
\multicolumn{1}{l|}{Mean}                         & \multicolumn{1}{l|}{9,43}            & 8,85           \\
\multicolumn{1}{l|}{Variance}                     & \multicolumn{1}{l|}{1,08}            & 0,94           \\
\multicolumn{1}{l|}{Observations}                 & \multicolumn{1}{l|}{15}              & 15             \\
\multicolumn{1}{l|}{Pearson Correlation}          & \multicolumn{1}{l|}{0,50}            &                \\
\multicolumn{1}{l|}{Hypothesized mean difference} & \multicolumn{1}{l|}{0}               &                \\
\multicolumn{1}{l|}{df}                           & \multicolumn{1}{l|}{14}              &                \\
\multicolumn{1}{l|}{t-statistic}                  & \multicolumn{1}{l|}{2,24}            &                \\
\multicolumn{1}{l|}{P(T\textless{}=t) one-tail}   & \multicolumn{1}{l|}{0,021}            &                \\
\multicolumn{1}{l|}{t-critical one-tail}          & \multicolumn{1}{l|}{1,76}            &                \\
\multicolumn{1}{l|}{P(T\textless{}=t) two-tail}   & \multicolumn{1}{l|}{0,042}            &                \\
\multicolumn{1}{l|}{t-critical two-tail}          & \multicolumn{1}{l|}{2,14}            &               
\end{tabular}
\caption{Paired t-test on the lateral position means of each participant in cm}
\label{table:tracking t-test}
\end{table}

\begin{figure}[h]
    \centering
	\includegraphics[width=1\linewidth]{images/boxplot_timeToFinish.png}
	\caption[
		Box plot time to finish
	]{
		Box plot of the times to finish the course in seconds.
	}
	\label{figure:boxplotTimeToFinish}
\end{figure}


\begin{table}[]
\centering
\begin{tabular}{lll}
\multicolumn{3}{l}{Time to finish in seconds - paired t-test}                                             \\
\multicolumn{1}{c|}{\textit{}}                    & \multicolumn{1}{l|}{\textit{normal}} & \textit{drunk} \\ \hline
\multicolumn{1}{l|}{Mean}                         & \multicolumn{1}{l|}{231,08}          & 243,62         \\
\multicolumn{1}{l|}{Variance}                     & \multicolumn{1}{l|}{521,24}          & 567,76         \\
\multicolumn{1}{l|}{Observations}                 & \multicolumn{1}{l|}{13}              & 13             \\
\multicolumn{1}{l|}{Pearson Correlation}          & \multicolumn{1}{l|}{0,64}            &                \\
\multicolumn{1}{l|}{Hypothesized mean difference} & \multicolumn{1}{l|}{0}               &                \\
\multicolumn{1}{l|}{df}                           & \multicolumn{1}{l|}{12}              &                \\
\multicolumn{1}{l|}{t-statistic}                  & \multicolumn{1}{l|}{-2,28}           &                \\
\multicolumn{1}{l|}{P(T\textless{}=t) one-tail}   & \multicolumn{1}{l|}{0,021}           &                \\
\multicolumn{1}{l|}{t-critical one-tail}          & \multicolumn{1}{l|}{1,78}            &                \\
\multicolumn{1}{l|}{P(T\textless{}=t) two-tail}   & \multicolumn{1}{l|}{0,042}           &                \\
\multicolumn{1}{l|}{t-critical two-tail}          & \multicolumn{1}{l|}{2,18}            &               
\end{tabular}
\caption{Paired t-test on the time to finish of each participant}
\label{table:timeToFinish t-test}
\end{table}


As the diagram in figure \ref{figure:boxplotTracking} shows, no participant experienced a major performance decrease in lateral positioning when driving with simulated drunkenness. 
On average the performance was slightly better when driving drunk than during the normal run.
The value deviations are normally distributed for the tracking data (W = 0.94; p = 0.397) and for the time to finish (W = 0.93; p = 0.307).
Conducting a paired t-test on the tracking data (Table \ref{table:tracking t-test}) shows that the p-value $<\alpha$ and therefore indicates a statistically significant set of data.
Taking the times to finish (Figure \ref{figure:boxplotTimeToFinish}) into consideration, 8 of the 13 participants that were able to finish the whole course, needed more time to finish the track when in the drunken state.
Since the length of the course was equal on each run it can be argued that some of the participants reduced their overall speed when drunk.
Other than that the plot in figure \ref{figure:boxplotTimeToFinish} does not show any major deviations from the recorded data. 
The paired t-test of the times to finish (Table \ref{table:timeToFinish t-test}) does validate the significance of the data.

\subsection{Reaction test}
\label{subsection:evaluation reaction}

Measuring and analysing the reaction times as a driving performance indicator is widely used as explained in section \ref{subsection:reaction time}.
The results of the reaction times are shown in figure \ref{figure:boxplotReactionTimes} and do not show a major difference in the drunk run.
The average reaction time of all participants is 0.98 seconds for the normal run and 1.03 for the drunk run.
An average increase of 0.05 seconds with simulated drunkenness is evaluated.
For each participant the average times range from 0.37 seconds to 1.68 seconds.
Deviations of the reaction times are normally distributed within the subjects (W = 0.96; p = 0.673) but the paired t-test of the data (Table \ref{table:reaction t-test}) determines a p-value $>\alpha$ and does not allow the rejection of the null hypothesis.


\begin{figure}[h]
    \centering
	\includegraphics[width=1\linewidth]{images/boxplot_reactionTimes.png}
	\caption[
		Box plot mean reaction times
	]{
		Box plot of the mean reaction times in seconds.
	}
	\label{figure:boxplotReactionTimes}
\end{figure}

\begin{table}[]
\centering
\begin{tabular}{lll}
\multicolumn{2}{l}{Mean reaction times in seconds - paired t-test}                       &                \\
\multicolumn{1}{c|}{\textit{}}                    & \multicolumn{1}{l|}{\textit{normal}} & \textit{drunk} \\ \hline
\multicolumn{1}{l|}{Mean}                         & \multicolumn{1}{l|}{0,98}            & 1,03           \\
\multicolumn{1}{l|}{Variance}                     & \multicolumn{1}{l|}{0,11}            & 0,09           \\
\multicolumn{1}{l|}{Observations}                 & \multicolumn{1}{l|}{15}              & 15             \\
\multicolumn{1}{l|}{Pearson Correlation}          & \multicolumn{1}{l|}{0,27}            &                \\
\multicolumn{1}{l|}{Hypothesized mean difference} & \multicolumn{1}{l|}{0}               &                \\
\multicolumn{1}{l|}{df}                           & \multicolumn{1}{l|}{14}              &                \\
\multicolumn{1}{l|}{t-statistic}                  & \multicolumn{1}{l|}{-0,49}           &                \\
\multicolumn{1}{l|}{P(T\textless{}=t) one-tail}   & \multicolumn{1}{l|}{0,317}           &                \\
\multicolumn{1}{l|}{t-critical one-tail}          & \multicolumn{1}{l|}{1,76}            &                \\
\multicolumn{1}{l|}{P(T\textless{}=t) two-tail}   & \multicolumn{1}{l|}{0,633}           &                \\
\multicolumn{1}{l|}{t-critical two-tail}          & \multicolumn{1}{l|}{2,14}            &               
\end{tabular}
\caption{Paired t-test on the reaction time means of each participant in seconds}
\label{table:reaction t-test}
\end{table}


\subsection{Simulator evaluation}
\label{subsection:simulator evaluation}

The first part of the questionnaire is aimed to gather feedback on the drunk driving simulator itself and identify any possibilities for improvement.
The second part includes questions on how the drunkenness in simulation is perceived and whether this simulation is a welcome addition to basic driver training.
\\
The handling of the car was perceived as acceptable (Mean 3.87; SD 0.96) but some participants were not quite satisfied on how the physical input acted in relation to a real car.
Pedals have to be pushed in very far for a decent effect and the steering wheel also has an unusually great dead range in the initial position before any steering input is detected.
Since the visual fidelity of the simulation is kept low for various reasons, the test subjects were asked on the importance of a realistic environment.
Almost everyone prefers authentic surroundings (Mean 4.26; SD 0.93) and wanted to see actual humans and traffic in the simulation.
The tasks that were part of the evaluation were perceived as reasonable and understandable (Mean 4.53; SD 0.5).
The simulated drunkenness was well accepted by most (Mean 4.2; SD 1.05), some did not really view the simulation to be a proper representation of reality.
\\
The overall perception of the drunk driving simulator was very positive.
Many claimed to have a better understanding on the effects of alcohol on their driving capabilities (Mean 4.34; SD 1.05) and the dangers involved with it (Mean 4.67; SD 0.79).
The importance of alcohol education in driver training was uniformly accepted (Mean 4.8; SD 0.4).
Therefore, the use of the simulation as an educational tool was perceived as reasonable (Mean 4.67; SD 0.6) and as a good addition to basic training (Mean 4.6; SD 0.71).
Most of the participants would use the simulation if it was available at their driving school (Mean 4.6; SD 1.02).

\subsection{Results}
\label{subsection:results}

With the evaluated data no significant negative impact of the drunk driving simulation on driver abilities was observed. 
The results of the tracking test did even indicate a very slight increase in ideal lateral positioning when drunk driving in the simulation.
A possible explanation is the increased time to finish the track, observed on the majority of participants.
The users had more time to focus on the tracking task and were potentially able to increase their performance in comparison to the normal run.
Since there is no recorded data on the speed or logged timestamps per measurement this is not more than an assumption.
The reaction test delivered a similar outcome of just a negligible increase in reaction times.
According to the paired t-test the recorded reaction times do not have a statistical significance.
A likely reason is that six measurements per run did not provide a sufficient amount of data.
The relative deviations of the reaction times demonstrate the learning effect over each run.
Since the participants of the group A started without any drunk effects, the subjects had an overall better reaction time in the drunk run.
This is very likely since the occurrence of the reaction test in the course is chosen in random order but not in random places i.e. in the second run the users are able to anticipate where the dummies could appear.
Overall, the measurement results do not reflect the assumed effect of the simulator.
Reaction times as well as the lateral positioning did not change significantly.
The recorded time to finish allows an extended interpretation of the lateral position, but the generic way of measuring the time renders further conclusions relatively vague.
\\
The results of the questionnaire indicate a very positive reception of the simulation.
Since driving under the influence of alcohol is a frequently discussed topic in driver training the intention of the simulation was well received and supported.
Many of the test subjects had experienced the effect of alcohol first hand despite their young age.
The quality of the implementation and of the input devices have room for improvements according to the study participants. 
Overall, the drunk driving simulator achieved its subjective goal in sensitizing the users to the negative effects of driving under the influence of alcohol and would likely be used if available in a driving school.

