The following section analyses scientific literature and statistics to show the need for the proposed drunk driving simulator as well as to provide a theoretical base to build a plausible simulation and to conduct a statistically sound user evaluation.
In section \ref{subsection:drunkDriving} the current state of drunk driving violations is viewed, while section \ref{subsection:alcoholEducation} identifies methods used to educate drivers on the impact of alcohol intoxication.
Section \ref{section:driver training} gathers information about simulations and their use in practice, section \ref{subsection:effects of alcohol} views the medical implications of alcohol on bodily functions and section \ref{section:measurePerformance} determines possible ways to measure the driver performance in a virtual simulation.


%
%###########################################
\subsection{Drunk Driving}
\label{subsection:drunkDriving}
%
%###########################################
Recent reports have shown that driving under the influence of alcohol is a major risk factor in many traffic accidents in Europe. \autocite{fell2014update}
\\
A frequently updated analysis on behalf of the Austrian Federal Ministry of work, social issues, health and consumer protection \autocite[107]{bmasgk2019alkohol} collects and evaluates alcohol-related statistics in Austria reaching back several decades.
Since 1998, when the criminal limit of the BAL was lowered from 0.8 to 0.5 in Austria, a major decrease in fines related to this offense as well as an increase in traffic controls was observed.
The authors conclude a significant decline in drunk drivers from this year on.
However, taking the data on traffic accidents with the involvement of alcohol intoxicated drivers into consideration, no substantial changes in accidents and injured persons is observed since 2012.
From 2012 to 2018 an average of 6.23\% (SD: 0.22\%) of all recorded traffic accidents with personal injuries were accidents with at least one participant being under the influence of alcohol.
In the same context an average 6.4\% (SD: 0.22\%) of injured persons were also involved in these types of accidents. 
\\
In a statistical report on traffic accidents in Germany of 2018 \autocite[]{destatis2019alkohol} a similar trend can be observed.
While the overall number of convicted drunk drivers decreased by 17.57\% from 2012 to 2017 \autocite[51]{destatis2019alkohol}, the involvement in traffic accidents in the same time period is constantly high.
It was determined that alcohol inebriated drivers were involved in 4.5\% of all accidents with personal injuries in 2018. 
7.5\% of all traffic fatalities were caused as a result of accidents where at least one participant was under the influence of alcohol.
According to this distribution alcohol induced traffic accidents tend to be much more severe.
The report further shows that 17.2\% of inebriated participants were between the age of 18 and 24 and 24.9\% were between 25 and 34 years old.
\\
In a collaborative study from 2019 of multiple Austrian institutes for traffic psychology \autocite[]{bartl2019alkohol}, 500 convicted drunk drivers were interviewed.
The study subjects were obligated to do a retraining course after they were caught driving under the influence of alcohol. 
The purpose of the retraining is to change the attitude towards drunk driving of each individual. 
33.5\% of the interviewees were involved in an accident. 
When asked what the person was thinking before getting into the car, 39.7\% believed that they were still able to drive.
The authors of the study also estimate that for each drunk driver that is caught during a police check, there are two inebriated drivers undetected.
\\
\textcite[1]{alonso2015driving} conducted a study to investigate the attitude towards driving under the influence of alcohol.
The results show that about a quarter of the interviewed participants admit to have driven drunk at least once.
The major motivation for this behaviour was the lack of alternative transportation and alcohol consumption related to meals. 
12.7\% of the interviewees with drunk driving experience on the other hand stated that they believe alcohol has no negative effects on their abilities.
%
%###########################################
\subsection{Alcohol education}
\label{subsection:alcoholEducation}
%
%###########################################

In the training of novice drivers, the theoretical implications of alcohol are taught and a fixed part in the teaching plan \footnote{http://www.bos.at/ebook/probe/fw/b/44/}.
Additionally, instructors tend to use graphic and shocking videos to visualize the possible outcomes of drunk driving.
There are also existing programs \footnote{http://www.close-to.net/}, in which convicted traffic offenders including drunk drivers visit driving schools and talk to students about their crimes and accidents.
Even though this makes novice drivers aware of the consequences, it does not convey the impact on their abilities even if just a small amount of alcohol is consumed.
Other methods to simulate alcohol intoxication are special googles that alter the vision of its users \footnote{https://www.fatalvision.com/}.
Certain tasks have to be performed while wearing these googles which are supposed to sensitize the users to the impairment caused by alcohol and other substances.
\\
As part of an project by a Chinese company \footnote{https://ifworlddesignguide.com/entry/246877-vr-drunk-driving-simulator} a similar VR simulation to the one in this work is developed.
This product aims to replicate a drunk drive to experience the negative effects of alcohol and is targeted for public use.
At the time of writing no further information to this project was found.


%
%###########################################
\subsection{Driver training with simulations}
\label{section:driver training}
%
%###########################################

\textcite[1]{vlakveld2005use} points out several advantages of a driver training simulation over driving lessons in real-life traffic situations.
The instructor has full control over the training conditions and is able to repeat and rehearse a given task as often as needed.
Dangerous situations can be practiced without any risk for the driver or his environment, which opens up the possibility to simulate scenarios that are very difficult to create in public traffic.
The trainees can also be provided with more detailed feedback since a computer-based simulation allows measuring of a wide variety of data e.g. speed, reaction times etc.
Even though these advantages enable the training of specific skills in isolated environments, it is uncertain if the learned skills can be applied in real traffic situations. 
According to \textcite[2]{vlakveld2005use} this is dependent on the quality of the simulation and a proper imitation of reality. 
The author suggests a combination with other training methods and devices to ensure the effectiveness of the simulator.
\\
The work of \textcite{waller1998transfer} confirms the importance of a realistic experience to achieve the best possible learning effect.
In theory, a perfect representation of a real-world environment in a training simulation would have the same learning impact as real-world training.
In case of a driving simulation this requires an authentic replication of a vehicle and its response to user input as well as a realistic surrounding area \autocite[130]{waller1998transfer}.
\\
A common issue observed in simulations, is the occurrence of the so-called simulation sickness \autocite[]{hettinger1992visually}.
Differences in perceived motion without actual physical movement may cause the affected person to experience a feeling of nausea. 
Another cause can be a significant delay between user input and the corresponding feedback coming from the simulation \autocite[2]{kemeny2014driving}.
To test the implications of simulation sickness on user enjoyment, \textcite{von2016cyber} developed and tested a VR racing game that tries to simulate the negative effects of alcohol on the human body by adding visual and auditive effects and distortion. 
Not contributing any scientific relevant findings, the authors conclude that the occurring sickness does not necessarily render a VR simulation useless.
\\
Depending on the requirements the driving simulator has to fulfil, some simplifications to the simulation have to be made to reduce the overall effort and costs during the development. 
These restrictions have further impact on the behaviour of its users but does not invalidate the use of simulators in driver training.  
Instead this requires instructors to carefully integrate the simulation into their training program. \autocite[3]{leitao1999evaluation} 
\\
In practice special equipment is available that is geared towards basic driver training and is mainly used in and by driving schools. 
An installation representing a car interior with all vital controls (pedals, gear box, steering wheel etc.) as well as multiple screens is available for purchase and has seen use in driving schools. \footnote{https://www.degener.de/produkt-kategorie/fahrschulunterricht/fahrsimulator/}
This simulation can be enhanced with a VR headset and several other modules to replicate difficult and dangerous situations in public traffic. 
\\


%
%###########################################
\subsection{Effects of alcohol intoxication}
\label{subsection:effects of alcohol}
%
%###########################################

In the following section a variety of research on the implications of ethanol intoxication is evaluated for the use in a virtual reality simulation. 
This includes especially changes in vision as well as general motor skills. 
For the purposes of this paper only a few characteristics of the influence of alcohol on human abilities can be simulated in a virtual environment. 
\\
\textbf{Reaction time} - It is known that the consumption of alcohol negatively affects the capability to process information and increases the reaction time (RT) \autocite[]{maylor1993alcohol}.
The implication on RT increases linearly with longer and/or more complex tasks.
\\
\textbf{Inattentional blindness} - A commonly observed phenomenon in scientific research as well as in practice is known as inattentional blindness \autocite[]{clifasefi2006blind}.
People tend to miss unexpected object or events in their field of vision when their attention is focused on something specific.
In a study held by \textcite{simons1999gorillas}, the participants were tasked to watch a video and count the number of passes in a basketball game. 
About the half of the participants failed to notice a man in a full gorilla costume that was also acting in the scene.
\textcite{clifasefi2006blind} evaluated the influence of alcohol on inattentional blindness with the same video as above and came to the conclusion that even small amounts increase the occurrence of this phenomenon. 
The authors link this outcome to a model named \textbf{alcohol myopia}.
It causes intoxicated individuals to focus more attention as usual to more prominent events and lose awareness for other details in their peripherals. \autocite{steele1990alcohol}
Since the participants were asked to perform a certain task (counting), the attention was drawn to the events regarding this task. 
In the context of driving a vehicle an example would be focusing on staying in lane and missing a pedestrian walking on the road.
\\
In a similar evaluation by \textcite{do2007effects} the control of visual attention under the influence of alcohol is observed. 
Participants were setup in an eye tracking system and were given tasks on where to focus their visual attention.
The difficulty of the tests varied and was divided into a simple and a more demanding task.
Results show that with a higher difficulty the subjects are less likely to be distracted by other events in their vision, a finding consistent with alcohol myopia. 
With a lower difficulty the attention tends to shift to other events and objects, the intoxicated subjects were less capable of ignoring distracting factors. 
The authors argue that this is related to the fact that alcohol decreases the ability to divide the attention in general \autocite{i1999effects}.
\\
\textbf{Vergence} - Alcohol is known to affect the vergence, the ability to move both eyes in opposite directions to achieve binocular vision \autocite{cassin1984dictionary}.
It reduces the limits in which the intoxicated person can fuse the perceived stereo images into a single three-dimensional image. 
In their study \textcite{miller1991effect} have shown that people under the influence of alcohol tend to experience diplopia for further away targets.
Diplopia is simultaneous perception of two images of the same object that are displaced vertically, horizontally, diagonally or rationally. \autocite[]{cassin1984dictionary}
A higher BAL will decrease the distance in which the person can see without diplopia for near and far targets.
For driving this might cause difficulties when trying to identify road signs and other traffic participants as well as trying to identify crucial information on the driver’s dashboard. 
The time required to fuse the perceived stereo images, also known as fusion latency, is increased for near and far targets. 
At intermediate distances the fusion latency was not affected by alcohol intoxication.
\\
\textbf{Eye movement} - The work of \textcite{moser1998effect} investigates the effect of alcohol on eye movement.
Smooth-pursuit eye movements, the ability to follow moving objects up to a certain velocity, are impaired for intoxicated people when tracking targets that are moving with a constant velocity. 
The decreased ability to pursuit movement is compensated by an increased number of catch-up \textbf{saccades} which are quick, simultaneous movements of both eyes between multiple fixation points \autocite{cassin1984dictionary}.
The velocity and latencies of saccadic eye movements is also decreased with a higher BAL.
There is no significant change in the accuracy of the saccades.
Intoxicated people show longer fixation duration and a reduced number of exploratory saccades when scanning a specific scenery.
The authors claim that the intoxication causes a deficit in understanding and processing the information of the fixated area.
\\
\textbf{Contrast sensitivity} - Even small doses of alcohol are known to reduce the contrast sensitivity for stationary and moving targets \autocite{nicholson1995effects}.
Contrast sensitivity can be described as the ability to distinguish lighter and darker areas in a perceived image.
\textcite[]{nicholson1995effects} mention that the impact on the visual performance on the contrast sensitivity was greater for moving targets. 
\\
\textbf{Balance} - A known effect of alcohol is a negative effect on the sense of balance \autocite{nieschalk1999effects}.
The intoxication tends to increase the postural sway and negatively affects the ability to coordinate the postural movement with voluntary activities. 
In their study, \textcite{nieschalk1999effects} find that with high BAL ($>$ 0.8‰) the body sway is increased significantly. 
With a lower percentage the sway is not as severe, but the authors mention that a stable posture is also influenced by psychological factors.
Since the test subject were aware of the aim of the study and were asked to perform to the best of their abilities, the results with low BAL might not reflect the actual impact. 
The authors also observed an improvement to the postural sway of certain subjects after consuming small doses of alcohol. 
\textcite{nieschalk1999effects} argue that this may be related to the stimulating effect that small doses of alcohol have on psychomotorical abilities and therefore improve overall performance.



%
%###########################################
\subsection{Measuring driver performance}
\label{section:measurePerformance}
%
%###########################################
Many personal factors, such as fatigue, stress, experience etc., influence every individual differently and are generally difficult to assess in a statistical manner. \autocite[16]{jurecki2011test}
In the following sections the possible ways to get objective measurements for driving performance are researched and evaluated.
\\

\subsubsection{Lateral position}
\label{subsection:lateral position}
A very commonly used metric to define driving behaviour is lateral positioning. \autocite[37]{ostlund2005driving}
It measures the driver’s ability to stay within a safe path and is influenced by many factors like driving experience, distraction and road environment.
It is common to use a straight road with multiple lanes while the user has to stay within the lane as precisely as possible \autocite[]{iso201026022} \autocite[]{brouwer1991divided}.
\\
A standardized test known as the "lane change test" (LCT) \autocite[]{iso201026022} is used to measure the impact on the drivers abilities when performing two tasks simultaneously.
The primary objective is to drive along a straight road with three lanes with a constant speed.
In regular intervals signs are appearing near the road and the users are instructed to change the lane according to the signal.
The vehicle properties are tracked with minimum sampling rate of 10 Hz during the experiment e.g. position on track, steer angle etc.
Even though this test is mainly used to measure the driver distraction by the secondary task, the path deviation measurement can still be used for the tracking test in this study.
As seen in figure \ref{figure:LCTNormative} this model can be used to identify the average deviation from the ideal path and is calculated as follows: \autocite[13]{iso201026022}
\begin{equation}
\label{equation:LCTMeanDeviation}
	\frac{1}{S}\sum[x_{deviation},i(\frac{y_{i+1}-y_{i-1}}{2})]
\end{equation}
\begin{equation}
\label{equation:deviation}
    x_{deviation,i} = |x_{position,i} - x_{reference,i}|
\end{equation}

\begin{figure}[h]
    \centering
	\includegraphics[width=1\linewidth]{images/lctNormativeModel.png}
	\caption[
		LCT normative model
	]{
		Standard analysis of LCT \autocite[77]{ostlund2005driving}
	}
	\label{figure:LCTNormative}
\end{figure}

The variables $x_{position,i}$ and $x_{reference,i}$ describe the lateral position of the vehicle on the track.
$x_{deviation,i}$ is the lateral deviation, $S$ the amount of data segments and $y$ the longitudinal position on the track.
\\

% The LCT is
% In a study to evaluate the differences in the driving capabilities of young and old drivers, \textcite[576]{brouwer1991divided} used a similar method as explained above.
% The subjects were tasked to continuously follow a straight road while a variable force pushed the vehicle to left or right.
% During the experiment a secondary assignment was given to the participants to determine the driving performance with divided attention.
% The amount of time the users were able to stay within their lane is described as "time on target". 
% To find the relative performance
% \\

\subsubsection{Reaction time}
\label{subsection:reaction time}

In the context of driving, reaction time is an important parameter in designing safe roads \autocite[195]{green2000long} and in the analysis of traffic accidents. \autocite[]{jurecki2011test}
\\
Even though there are established norms on the average reaction times, many estimates in known literature are differing significantly from the standardized values. \autocite[]{green2000long} 
Literature reviews regarding the reaction times of driver’s state, that the main reason behind these discrepancies are the different testing conditions used by the researchers. \autocite[]{green2000long} \autocite[]{jurecki2011test}
\\
In their work \textcite[]{jurecki2011test} review the methods for determining reaction times while driving a vehicle.
Many of the analysed studies use simple cues like single lights or tones as a signal for the driver to react to.
The author states that in real traffic situations the indicators on when the driver should e.g. brake are much more complex.
In most cases additional actions are required to e.g. avoid a crash than just braking which further impacts the driver’s behaviour.
More realistic obstacles like inflatable car dummies \autocite[203]{jansson2002decision} try to simulate complex stimuli and are more popular in recent study reports. \autocite[17]{jurecki2011test}
It has to be noted that many of the above-mentioned tests were not held in virtual driving simulations but on test tracks or real roads.
As discussed in section \ref{section:driver training}, the use of a virtual environment allows for a more authentic scenario to evaluate the reaction times of the user as well as for an easy way to get exact measurements.
\\
For a better understanding reaction time can be separated into multiple components. \autocite[]{boff1988engineering}.
The time needed for perceiving a signal or object and deciding on a response is known as the "mental processing time".
In the proposed simulator this is the time the user needs for spotting the dummy, realizing it is moving towards the road and deciding to brake to avoid a collision.
The time it takes for the driver to move his foot away from the accelerator and pushing the brake pedal is known as "movement time".
The "device response time" is the amount of time the vehicle requires to perform the requested action e.g. the time it takes to fully stop the car.
\\
As \textcite[200]{green2000long} states that the mental processing time is the truest form of reaction time but many studies include the other mentioned components in their estimations.
Since the mental processing time is quite difficult to measure objectively without a physical response the test setup in this work also includes the movement time when evaluating the reaction time.
The device response time will not be included in the measurement because it does not contribute any valuable information to the purposes of this study.

