Looking at recent statistics of traffic violations and accidents with the involvement of alcohol intoxication, drunk driving is a persistent problem.
Through various educational and legal measures, the number of drunk drivers was reduced significantly, but alcohol-related traffic accidents and injuries are still at a constant level.
Since young adults from the age of 18 to 34 are most commonly involved in alcohol-related accidents, proper alcohol education has to be applied as early as possible.
Many methods for this purpose are utilised in driver education and aim to sensitize the novice drivers for the dangers and consequences of drunk driving.
Since VR driving simulations are already tested and applied to be an addition to driver training, a VR drunk driving simulator is a potential method to improve the alcohol education of young drivers.
An immersive and engaging VR environment allows the users to experience the impacts of alcohol in a realistic scenario without any risks.
This should ideally sensitize the users to not drive under the influence of alcohol and decrease the amount of drunk driving violations of young adults especially.
The idea and implementation of a VR drunk driving simulator for public use was already established by a Chinese company in 2018 \footnote{https://ifworlddesignguide.com/entry/246877-vr-drunk-driving-simulator}.
No scientific research or material to this project was found to the time of writing and no other related work covering this topic is available.
A proper evaluation of such a simulation is needed to validate the effectiveness as an education tool.
\\
Even though the evaluation of the drunk driving simulator was well received by the study participants, it was not successful in statistically validating the intended negative implications on the driver’s abilities.
Since decades of research prove that alcohol significantly decreases the performance of drivers, the cause for the result of this work lies in the quality of the simulation and its evaluation.
Using a VR environment and an HMD allows for an immersive experience which is desirable for any simulation.
In the case of a VR driving simulation a major issue is the movement of the vehicle that continuously changes the position and rotation of the user view.
During the pre-tests and the study many persons without any VR experience tested the simulator.
In the first few minutes of driving almost everybody was overwhelmed by the visual immersion but the lack of any physical feedback.
This resulted in users pulling on the steering wheel until the clamps used for attaching the wheel to the desk became loose and the steering wheel detached from the desk (No equipment was damaged).
Also, a first prototype in a very dense town with many intersections and sharp turns was very difficult to navigate for an extended period of time due to very rapidly occurring simulation sickness.
With the addition of the drunk effects the simulation became almost unbearable. 
As described in section \ref{subsection:virtual environment}, a good compromise of an authentic environment and easy navigation was established and simulator sickness occurred less frequent. 
The missing physical feedback was still a nuisance for many but was not possible to reproduce without advanced equipment.
Highly developed simulators use platforms \footnote{https://www.avl.com/web/guest/-/driving-simulator} \footnote{https://dofreality.com/} to achieve motion feedback with up to six degrees of freedom and are used in racing and flight simulations.
Even though this would enhance the quality of the simulation greatly, major downsides are the price, space requirement and static nature of the apparatus. 
\\
Another area that could be improved by motion feedback is the simulated drunkenness.
This work solely focuses on vision-based effects of alcohol even though many more human senses and skills are influenced by alcohol intoxication. 
E.g. negative impacts on balance could be realized by manipulating the motion platform to subtly shift the user in random directions.
During prototyping, ways to artificially recreate implications on reaction times and balance were tested.
Shifting or rotating the camera to replicate impacts on balance was very unpleasant in VR and scrapped immediately.
Input delays on the wheel and pedals for decreased reaction times rendered the car almost uncontrollable.
It also would have manipulated all the measurements in this work and artificially created deviations in lateral position and reaction times. 
A random shift of the car to the left or right was tested but discarded because it would also distort the tracking measurements.
In hindsight implementing the mentioned effects would have definitely improved the educational message since some of the study participants did not feel drunk in the simulation.
\\
An interesting addition would be the use of eye tracking technology in VR. \autocite[]{Clay_König_König_2019}
Knowing the gaze of the user opens up many more possibilities to manipulate the view based on this information. 
Instead of linking the severity of the visual effects to the movement of the HMD, the movement of the gaze could be used for this purpose.
The negative impacts on vergence, the ability to focus on near or far targets without seeing "double", could also be implemented realistically with this technology.
\\
Another quality improvement to the simulation would be a redesign of the virtual surrounding area.
To improve the performance and allow for more visual fidelity, the novel Unity Universal Render Pipeline (URP) could be utilised \footnote{https://docs.unity3d.com/Packages/com.unity.render-pipelines.universal@8.2/manual/index.html}.
The rendering can be adjusted via C# script and potentially be optimized for the use of the drunk screen shaders in VR.
During development a switch to the URP was tested but it created several issues with corrupted materials.
To use this technology more development time would have been required and it is very likely that the project had to be rebuilt from scratch to make it work flawlessly.
The drunk shaders would also need a rework to work with rendering of the different pipeline.
\\
Other than graphical improvements, the virtual environment could be extended by a more authentic traffic situation.
Pedestrians, ongoing traffic, crossings, traffic signs etc. would have raised the immersion and difficulty greatly, but were not implemented because of limited development time, higher risk of simulator sickness and performance concerns.
These additional distraction factors would have also impacted the measurement tests in this work.
Reliable and exact measurements would have been much more complex to implement and evaluate.
\\
Since the results of the reaction test were not statistically significant, a rework of the whole setup is required.
The amount of measurements was not sufficient for a significant outcome and the reaction test has to be performed differently or much more frequently.
Statistically evaluating the driver performance could have been much more effective in an isolated scene specifically made to measure driver behaviour, similar to the LCT test discussed in section \ref{section:measurePerformance}.
This could potentially be combined with a different scene resembling a real traffic environment, where the users are given verbal commands by the test coordinator.
With the assist of a qualified expert i.e. a driving instructor or traffic psychologist, the behaviour of the user could be observed and analysed during the study.
Both of these evaluations, a statistically sound performance measurement and expert observation of driving behaviour, are likely to improve the quality of the collected data.
\\
