Driving under the influence of alcohol is a constantly observed traffic violation and a contributing factor for many traffic accidents in Europe \autocite[]{fell2014update}.
This behaviour is especially common for adolescents \autocite[]{arnett1990drunk} and is an integral part in the theoretical teaching plan in Austrian driving schools  \footnote{http://www.bos.at/ebook/probe/fw/b/44/}.
This includes basic information on how the blood alcohol level (BAL) is affected, the common effects of alcohol on the driver and the legal framework.
Considering the data on traffic violations in Austria regarding drunk drivers \autocite[107]{bmasgk2019alkohol} a decline in criminal charges is observed in recent years.
This improvement can be linked to more frequent traffic controls and stricter legal limits for the BAL. 
Also, educators are utilising different methods, e.g. shocking imagery and stories \footnote{http://www.close-to.net/} or specifically designed googles \footnote{https://www.fatalvision.com/}, to sensitize people to the implications of alcohol.
Nevertheless, a significant number of people are known to have driven under the influence of alcohol, with most of them being well aware on the implications on their abilities and possible consequences of their actions \autocite[]{alonso2015driving}.
To this date it is still of high relevance to properly educate drivers about the dangers of drunk driving and find new ways to impart this knowledge, especially to novice drivers.
\\
With the advance of digital simulations and their acceptance as serious learning tools \autocite[]{backlund2006computer}, many advantages to real-life training can be determined. 
Illegal or dangerous traffic situations can be reproduced and trained multiple times and are known to positively impact the learning effect when executed right \autocite[]{vlakveld2005use}.
Using Virtual Reality (VR), the immersion of driving simulators can be enhanced significantly \autocite[498]{ihemedu2015development}, which is argued to be an important factor in the quality of a simulation designed for training. \autocite[2]{vlakveld2005use}
The improved effectiveness of VR based learning environments is evaluated by \textcite{chen2006design} and is argued to be better than conventional non-VR learning software. 
An implementation of a VR driving simulation that replicates an alcohol intoxicated car ride was tested by \textcite[]{von2016cyber} and could potentially be adopted to use in driver education.
Without any major risk this would allow educators to teach the implications of alcohol on driver abilities in an engaging way. 
\\
Aim of this work is to develop such a drunk driving simulation in VR and to evaluate its potential use in basic driver training.
The drunk driving simulator is supposed to severely impact the driver abilities and create a lasting experience to sensitize the user to the negative implications of driving drunk.
Geared towards the use in driving schools with minimal cost and space requirements, commonly available equipment and technology is used for the development. 
Possible ways to create an authentic replication of alcohol intoxication are explored and tested in a laboratory experiment.
The study is held at an Austrian driving school with 15 novice drivers as test subjects.
Three different sets of objective measurements are supposed to identify any relative deviations in driver behaviour when driving in the same simulator with and without the proposed simulated drunkenness.
These measurements include the reaction time, the ability to stay within a given path and the overall time to finish the car drive.
This data is then statistically evaluated to potentially validate the negative effect of the drunk driving simulation on driver abilities.
While reaction time is expected to increase and viewed separately, the ability to stay within a path, which is supposed to deteriorate, and the time to finish is evaluated together to find potential relations. 
A questionnaire after the experiment is conducted to gather feedback on the simulation as well as to subjectively determine the acceptance as a serious learning tool in basic driver training.
Finally, the outcome of the study is discussed and an outlook for future work is given. 



