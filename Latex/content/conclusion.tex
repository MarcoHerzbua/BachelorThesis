This work presents the development and evaluation of a VR simulation aimed to replicate a car drive under the influence of alcohol.
It is aimed to be used in basic driver training and sensitize young drivers to the negative implications of alcohol intoxication while driving.
Since alcohol-related traffic violations are still a constant problem, the simulation is intended to be another method to further reduce the number of drunk drivers.
During the user study a decrease in driver performance when using the simulation was not statistically evaluated.
However, the majority of the participants claimed to be influenced by the educational message of the simulator and approved of the potential use in driver training.
With improvements to the implementation and further evaluations, the simulation could be used in driving schools as an voluntary addition to the common teaching plans. 
