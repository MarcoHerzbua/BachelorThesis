Alcohol-related traffic violations and accidents are a constant problem in public traffic and drunk driving is a frequently discussed topic in basic driver training.
Various methods to sensitize people to the negative implications of driving under the influence of alcohol exist and managed to decrease the legal violations by a significant amount over the last years.
However, research shows that especially young adults tend to underestimate the dangers of drunk driving and more effective ways to impart the educational message are needed.
With virtual simulations being an accepted alternative to real-life driver training, this work proposes a virtual reality driving simulation that replicates alcohol intoxication to the user.
The aim of the simulator is to give novice drivers an understanding of the severe impact of alcohol on their driving capabilities.
Intended to be used as an addition to basic driver training, a study was conducted at a driving school with 15 participants, all of which were in training to get their driving license at that point.
Each subject was tasked to fulfil the same set of objectives two times, once with simulated drunkenness and once without any effects.
To measure the impact on the driving performance, the reaction time, the ability to stay within a given path and the time it took to finish the given tasks was recorded.
Additionally, a questionnaire was handed out to gather the subjective perception and opinion of the simulation.
The evaluation determined the relative performance changes of each subject and analysed the results of the questionnaire.
Since no significant deviations from the normal to the drunk run were observed, the effectiveness of the simulation could not be validated by the measured data.
The results of the questionnaire show a positive reception of the simulator and its educational intention.
With improvements on the quality and further evaluation, the use of a virtual reality drunk driving simulation could be a potential voluntary addition to basic driver training programs.
