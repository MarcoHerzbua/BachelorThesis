Verkehrsunfälle und -vergehen im Zusammenhang mit Alkohol sind ein konstantes Problem im öffentlichen Straßenverkehr und Trunkenheit am Steuer wird häufig in der Fahrausbildung diskutiert.
Eine Reihe von Methoden versuchen die Menschen für die negativen Auswirkungen von Alkohol am Steuer zu sensibilisieren und schafften es diese Art von Verkehrsvergehen zu senken.
Trotzdem zeigt die Forschung, dass vor allem junge Erwachsene die Gefahren von Alkohol am Steuer unterschätzen und bessere Aufklärungsmaßnahmen benötigt werden. 
Mit der vermehrten Anwendung von virtuellen Simulationen in der Fahrerausbildung schlägt diese Arbeit einen Virtual Reality Fahrsimulator vor, der eine betrunkene Autofahrt simulieren soll.
Damit sollen Fahranfänger auf die Beeinträchtigung ihrer Fähigkeiten durch Alkohol aufmerksam gemacht werden und es soll als zusätzliches Werkzeug in der Führerscheinausbildung agieren.
Dazu wurde eine Studie in einer Fahrschule mit insgesamt 15 Fahrschülern durchgeführt.
Jeder Einzelne musste eine Sammlung von Aufgaben insgesamt zwei Mal durchführen, einmal mit simulierter Trunkenheit und einmal ohne zusätzliche Einschränkungen.
Die Auswirkung auf die Fahrfähigkeiten wurde durch die Reaktionszeit, die Spurhaltung und den Zeitaufwand für die Aufgaben gemessen.
Zusätzlich mussten die Teilnehmer einen Fragebogen ausfüllen, der die subjektive Meinung und Wahrnehmung feststellen sollte.
Die Auswertung beschäftigte sich mit den relativen Unterschieden jedes Teilnehmers sowie mit den Ergebnissen des Fragebogens.
Da keine relevanten Abweichungen von der normalen zur betrunkenen Fahrt festgestellt werden konnten, ist die Effektivität des Simulators nicht objektiv bestätigt.
Die Ergebnisse des Fragebogens zeigen hingegen eine positive Einstellung zur Simulation und zu dessen Bemühungen zur Aufklärung von Fahranfängern.  
Mit Qualitätsverbesserungen und weiteren Evaluierungen könnte der Fahrsimulator im Fahrschulunterricht als eine freiwillige Zusatzleistung verwendet werden.
